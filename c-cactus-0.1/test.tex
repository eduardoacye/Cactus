\documentclass[letterpaper, 12pt]{article}

\usepackage[spanish]{babel}
\usepackage[utf8]{inputenc}

\usepackage[margin=1in]{geometry}
\usepackage[sc]{mathpazo}
\usepackage[T1]{fontenc}
\usepackage{bm}

\linespread{1.05}
\setlength{\parindent}{0cm}

\begin{document}

\hfill {\LARGE \textbf{Programa en C++} }

\bigskip


\hfill {\large \textbf{Eduardo Acuña Yeomans} }

\bigskip


\bigskip

Este programa es una demostración para el programa c-cactus
derivado de cactus (generador de código \LaTeX  a partir de código en
Scheme).

\bigskip


\begin{verbatim}
#include <iostream>
\end{verbatim}


\begin{verbatim}
using namespace std;
\end{verbatim}


\bigskip

\bigskip

\textbf{f :} \emph{Real} $\boldsymbol{\rightarrow}$ \emph{Real}


Function defined as $f(x)=x^2+3x-\frac{x}{0.23}+9.2$.
\begin{verbatim}
double f(double x)
{
    return x*x + 3x - x/.23 + 9.2;
}
\end{verbatim}


\bigskip

\bigskip

\textbf{main :} \emph{Integer} $\times$ \emph{String} $\boldsymbol{\rightarrow}$ \emph{Integer}


Un `holamundo` común y corriente.
\begin{verbatim}
int main(int argc, char ** argv)
{
    cout << "Hola mundo";
\end{verbatim}


\begin{verbatim}
    // Imprime en pantalla el valor de f evaluado en 2.
    cout << f(2) << endl;
\end{verbatim}


\begin{verbatim}
    return 0;
}
\end{verbatim}




\end{document}